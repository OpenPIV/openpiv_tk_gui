%% Generated by Sphinx.
\def\sphinxdocclass{report}
\documentclass[letterpaper,10pt,english]{sphinxmanual}
\ifdefined\pdfpxdimen
   \let\sphinxpxdimen\pdfpxdimen\else\newdimen\sphinxpxdimen
\fi \sphinxpxdimen=.75bp\relax

\PassOptionsToPackage{warn}{textcomp}
\usepackage[utf8]{inputenc}
\ifdefined\DeclareUnicodeCharacter
% support both utf8 and utf8x syntaxes
  \ifdefined\DeclareUnicodeCharacterAsOptional
    \def\sphinxDUC#1{\DeclareUnicodeCharacter{"#1}}
  \else
    \let\sphinxDUC\DeclareUnicodeCharacter
  \fi
  \sphinxDUC{00A0}{\nobreakspace}
  \sphinxDUC{2500}{\sphinxunichar{2500}}
  \sphinxDUC{2502}{\sphinxunichar{2502}}
  \sphinxDUC{2514}{\sphinxunichar{2514}}
  \sphinxDUC{251C}{\sphinxunichar{251C}}
  \sphinxDUC{2572}{\textbackslash}
\fi
\usepackage{cmap}
\usepackage[T1]{fontenc}
\usepackage{amsmath,amssymb,amstext}
\usepackage{babel}



\usepackage{times}
\expandafter\ifx\csname T@LGR\endcsname\relax
\else
% LGR was declared as font encoding
  \substitutefont{LGR}{\rmdefault}{cmr}
  \substitutefont{LGR}{\sfdefault}{cmss}
  \substitutefont{LGR}{\ttdefault}{cmtt}
\fi
\expandafter\ifx\csname T@X2\endcsname\relax
  \expandafter\ifx\csname T@T2A\endcsname\relax
  \else
  % T2A was declared as font encoding
    \substitutefont{T2A}{\rmdefault}{cmr}
    \substitutefont{T2A}{\sfdefault}{cmss}
    \substitutefont{T2A}{\ttdefault}{cmtt}
  \fi
\else
% X2 was declared as font encoding
  \substitutefont{X2}{\rmdefault}{cmr}
  \substitutefont{X2}{\sfdefault}{cmss}
  \substitutefont{X2}{\ttdefault}{cmtt}
\fi


\usepackage[Bjarne]{fncychap}
\usepackage{sphinx}

\fvset{fontsize=\small}
\usepackage{geometry}


% Include hyperref last.
\usepackage{hyperref}
% Fix anchor placement for figures with captions.
\usepackage{hypcap}% it must be loaded after hyperref.
% Set up styles of URL: it should be placed after hyperref.
\urlstyle{same}


\usepackage{sphinxmessages}
\setcounter{tocdepth}{1}



\title{OpenPivGui}
\date{Sep 25, 2020}
\release{0.2.9}
\author{OpenPivGui Community}
\newcommand{\sphinxlogo}{\vbox{}}
\renewcommand{\releasename}{Release}
\makeindex
\begin{document}

\ifdefined\shorthandoff
  \ifnum\catcode`\=\string=\active\shorthandoff{=}\fi
  \ifnum\catcode`\"=\active\shorthandoff{"}\fi
\fi

\pagestyle{empty}
\sphinxmaketitle
\pagestyle{plain}
\sphinxtableofcontents
\pagestyle{normal}
\phantomsection\label{\detokenize{index::doc}}


OpenPivGui is a graphical user interface, providing an efficient workflow for evaluating and postprocessing particle image velocimetry (PIV) images. OpenPivGui relies on the Python libraries provided by the \sphinxhref{http://www.openpiv.net/}{OpenPIV project}.

\begin{figure}[htbp]
\centering
\capstart

\noindent\sphinxincludegraphics{{open_piv_gui_vector_plot}.png}
\caption{OpenPivGui.}\label{\detokenize{index:id1}}\end{figure}


\chapter{Installation}
\label{\detokenize{installation:installation}}\label{\detokenize{installation::doc}}

\section{Python Package Index}
\label{\detokenize{installation:python-package-index}}
You may use \sphinxcode{\sphinxupquote{pip}} to install OpenPivGui:

\begin{sphinxVerbatim}[commandchars=\\\{\}]
\PYG{n}{pip3} \PYG{n}{install} \PYG{n}{openpivgui}
\end{sphinxVerbatim}


\section{Launching}
\label{\detokenize{installation:launching}}
Launch OpenPivGui by executing:

\begin{sphinxVerbatim}[commandchars=\\\{\}]
\PYG{n}{python3} \PYG{o}{\PYGZhy{}}\PYG{n}{m} \PYG{n}{openpivgui}\PYG{o}{.}\PYG{n}{OpenPivGui}
\end{sphinxVerbatim}


\chapter{Usage}
\label{\detokenize{usage:usage}}\label{\detokenize{usage::doc}}

\section{Video Tutorial}
\label{\detokenize{usage:video-tutorial}}
Learn how to use and extend OpenPivGui in less than eight minutes in a \sphinxhref{https://video.fh-muenster.de/Panopto/Pages/Viewer.aspx?id=309dccc2-af58-44e0-8cd3-ab9500c5b7f4}{video tutorial}.


\section{Workflow}
\label{\detokenize{usage:workflow}}\begin{enumerate}
\sphinxsetlistlabels{\arabic}{enumi}{enumii}{}{.}%
\item {} 
Press the button »open directory« and choose a directory that contains some PIV images.

\item {} 
To inspect the images, click on the links in the file\sphinxhyphen{}list on the right side of the OpenPivGui window.

\item {} 
Walk through the riders, select the desired functions, and edit the corresponding parameters.

\item {} 
Press »start processing« to start the evaluation.

\item {} 
Inspect the results by clicking on the links in the file\sphinxhyphen{}list.

\item {} 
Use the »back« and »forward« buttons to inspect intermediate results. Use the »back« and »forward« buttons also to list the image files again, and to repeat the evaluation.

\item {} 
Use »dump settings« to document your project. You can recall the settings anytime by pressing »load settings«. The lab\sphinxhyphen{}book entries are also restored from the settings file.

\end{enumerate}


\section{Adaption}
\label{\detokenize{usage:adaption}}
First, get the source code. There are two possibilities:
\begin{enumerate}
\sphinxsetlistlabels{\arabic}{enumi}{enumii}{}{.}%
\item {} 
Clone the git repository:

\begin{sphinxVerbatim}[commandchars=\\\{\}]
\PYG{n}{git} \PYG{n}{clone} \PYG{n}{https}\PYG{p}{:}\PYG{o}{/}\PYG{o}{/}\PYG{n}{github}\PYG{o}{.}\PYG{n}{com}\PYG{o}{/}\PYG{n}{OpenPIV}\PYG{o}{/}\PYG{n}{openpiv\PYGZus{}tk\PYGZus{}gui}\PYG{o}{.}\PYG{n}{git}
\end{sphinxVerbatim}

\item {} 
Find out, where pip3 placed the source scripts and edit them in place:

\begin{sphinxVerbatim}[commandchars=\\\{\}]
\PYG{n}{pip3} \PYG{n}{show} \PYG{n}{openpivgui}
\end{sphinxVerbatim}

\end{enumerate}

In both cases, cd into the subdirectory \sphinxcode{\sphinxupquote{openpivgui}} and find the main scripts to edit:
\begin{itemize}
\item {} 
\sphinxcode{\sphinxupquote{OpenPivParams.py}}

\item {} 
\sphinxcode{\sphinxupquote{OpenPivGui.py}}

\end{itemize}

Usually, there are two things to do:
\begin{enumerate}
\sphinxsetlistlabels{\arabic}{enumi}{enumii}{}{.}%
\item {} 
Adding new variables and a corresponding widgets to enable users to modify its values.

\item {} 
Adding a new method (function).

\end{enumerate}


\subsection{Adding new variables}
\label{\detokenize{usage:adding-new-variables}}
Open the script \sphinxcode{\sphinxupquote{OpenPivParams.py}}. Find the method \sphinxcode{\sphinxupquote{\_\_init\_\_()}}. There, you find a variable, called \sphinxcode{\sphinxupquote{default}} of type dict. All widgets like checkboxes, text entries, and option menus are created based on the content of this dictionary.

By adding a dictionary element, you add a variable. A corresponding widget is automatically created. Example:

\begin{sphinxVerbatim}[commandchars=\\\{\}]
\PYG{l+s+s1}{\PYGZsq{}}\PYG{l+s+s1}{corr\PYGZus{}window}\PYG{l+s+s1}{\PYGZsq{}}\PYG{p}{:}             \PYG{c+c1}{\PYGZsh{} key}
     \PYG{p}{[}\PYG{l+m+mi}{3020}\PYG{p}{,}                \PYG{c+c1}{\PYGZsh{} index}
     \PYG{l+s+s1}{\PYGZsq{}}\PYG{l+s+s1}{int}\PYG{l+s+s1}{\PYGZsq{}}\PYG{p}{,}                \PYG{c+c1}{\PYGZsh{} type}
     \PYG{l+m+mi}{32}\PYG{p}{,}                   \PYG{c+c1}{\PYGZsh{} value}
     \PYG{p}{(}\PYG{l+m+mi}{8}\PYG{p}{,} \PYG{l+m+mi}{16}\PYG{p}{,} \PYG{l+m+mi}{32}\PYG{p}{,} \PYG{l+m+mi}{64}\PYG{p}{,} \PYG{l+m+mi}{128}\PYG{p}{)}\PYG{p}{,} \PYG{c+c1}{\PYGZsh{} hints}
     \PYG{l+s+s1}{\PYGZsq{}}\PYG{l+s+s1}{window size}\PYG{l+s+s1}{\PYGZsq{}}\PYG{p}{,}        \PYG{c+c1}{\PYGZsh{} label}
     \PYG{l+s+s1}{\PYGZsq{}}\PYG{l+s+s1}{Size in pixel.}\PYG{l+s+s1}{\PYGZsq{}}\PYG{p}{]}\PYG{p}{,}    \PYG{c+c1}{\PYGZsh{} help}
\end{sphinxVerbatim}

In \sphinxcode{\sphinxupquote{OpenPivGui.py}}, you access the value of this variable via \sphinxcode{\sphinxupquote{p{[}\textquotesingle{}corr\_window\textquotesingle{}{]}}}. Here, \sphinxcode{\sphinxupquote{p}} is the instance name of an \sphinxcode{\sphinxupquote{OpenPivParams}} object. Typing:

\begin{sphinxVerbatim}[commandchars=\\\{\}]
\PYG{n+nb}{print}\PYG{p}{(}\PYG{n+nb+bp}{self}\PYG{o}{.}\PYG{n}{p}\PYG{p}{[}\PYG{l+s+s1}{\PYGZsq{}}\PYG{l+s+s1}{corr\PYGZus{}window}\PYG{l+s+s1}{\PYGZsq{}}\PYG{p}{]}\PYG{p}{)}
\end{sphinxVerbatim}

will thus result in:

\begin{sphinxVerbatim}[commandchars=\\\{\}]
\PYG{l+m+mi}{32}
\end{sphinxVerbatim}

The other fields are used for widget creation:
\begin{itemize}
\item {} 
index: An index of 3xxx will place the widget on the third rider (»PIV«).

\item {} \begin{description}
\item[{type:}] \leavevmode\begin{itemize}
\item {} \begin{description}
\item[{\sphinxcode{\sphinxupquote{None}}: Creates a new notebook rider.}] \leavevmode\begin{itemize}
\item {} 
\sphinxcode{\sphinxupquote{bool}}: A checkbox will be created.

\item {} 
\sphinxcode{\sphinxupquote{str{[}{]}}}: Creates a listbox.

\item {} 
\sphinxcode{\sphinxupquote{text}}: Provides a text area.

\item {} 
\sphinxcode{\sphinxupquote{float}}, \sphinxcode{\sphinxupquote{int}}, \sphinxcode{\sphinxupquote{str}}: An entry widget will be created.

\end{itemize}

\end{description}

\end{itemize}

\end{description}

\item {} 
hints: If hints is not \sphinxcode{\sphinxupquote{None}}, an option menu is provided with \sphinxcode{\sphinxupquote{hints}} (tuple) as options.

\item {} 
label: The label next to the manipulation widget.

\item {} 
help: The content of this field will pop up as a tooltip, when the mouse is moved over the widget.

\end{itemize}


\subsection{Adding a new method}
\label{\detokenize{usage:adding-a-new-method}}
Open the script \sphinxcode{\sphinxupquote{OpenPivGui}}. There are two main possibilities, of doing something with the newly created variables:
\begin{enumerate}
\sphinxsetlistlabels{\arabic}{enumi}{enumii}{}{.}%
\item {} 
Extend the existing processing chain.

\item {} 
Create a new method.

\end{enumerate}


\subsubsection{Extend existing processing chain}
\label{\detokenize{usage:extend-existing-processing-chain}}
Find the function definition \sphinxcode{\sphinxupquote{start\_processing()}}. Add another \sphinxcode{\sphinxupquote{if}} statement and some useful code.


\subsubsection{Create a new method}
\label{\detokenize{usage:create-a-new-method}}
Find the function definition \sphinxcode{\sphinxupquote{\_\_init\_buttons()}}. Add something like:

\begin{sphinxVerbatim}[commandchars=\\\{\}]
\PYG{n}{ttk}\PYG{o}{.}\PYG{n}{Button}\PYG{p}{(}\PYG{n}{f}\PYG{p}{,}
           \PYG{n}{text}\PYG{o}{=}\PYG{l+s+s1}{\PYGZsq{}}\PYG{l+s+s1}{button label}\PYG{l+s+s1}{\PYGZsq{}}\PYG{p}{,}
           \PYG{n}{command}\PYG{o}{=}\PYG{n+nb+bp}{self}\PYG{o}{.}\PYG{n}{new\PYGZus{}func}\PYG{p}{)}\PYG{o}{.}\PYG{n}{pack}\PYG{p}{(}
                   \PYG{n}{side}\PYG{o}{=}\PYG{l+s+s1}{\PYGZsq{}}\PYG{l+s+s1}{left}\PYG{l+s+s1}{\PYGZsq{}}\PYG{p}{,} \PYG{n}{fill}\PYG{o}{=}\PYG{l+s+s1}{\PYGZsq{}}\PYG{l+s+s1}{x}\PYG{l+s+s1}{\PYGZsq{}}\PYG{p}{)}
\end{sphinxVerbatim}

Add the new function:

\begin{sphinxVerbatim}[commandchars=\\\{\}]
\PYG{k}{def} \PYG{n+nf}{new\PYGZus{}func}\PYG{p}{(}\PYG{n+nb+bp}{self}\PYG{p}{)}\PYG{p}{:}
    \PYG{c+c1}{\PYGZsh{} do something useful here}
    \PYG{k}{pass}
\end{sphinxVerbatim}


\subsection{Testing}
\label{\detokenize{usage:testing}}
Overwrite the original scripts in the installation directory (locate the installation directory by \sphinxcode{\sphinxupquote{pip3 show openpivgui}}) with your altered version and test it. There are test images in the \sphinxhref{https://github.com/OpenPIV/openpiv\_tk\_gui/tree/master/tst\_img}{OpenPivGui Github repository}, if needed.


\section{Reusing code}
\label{\detokenize{usage:reusing-code}}
The openpivgui modules and classes can be used independently from the GUI. The can be used in other scipts or jupyter notebooks and some can be called from the command line directly.


\section{Troubleshooting}
\label{\detokenize{usage:troubleshooting}}\begin{description}
\item[{I can not install OpenPivGui.}] \leavevmode
Try \sphinxcode{\sphinxupquote{pip}} instead of \sphinxcode{\sphinxupquote{pip3}} or try the \sphinxcode{\sphinxupquote{\sphinxhyphen{}\sphinxhyphen{}user}} option:

\begin{sphinxVerbatim}[commandchars=\\\{\}]
\PYG{n}{pip} \PYG{n}{install} \PYG{o}{\PYGZhy{}}\PYG{o}{\PYGZhy{}}\PYG{n}{user} \PYG{n}{openpivgui}
\end{sphinxVerbatim}

Did you read the error messages? If there are complaints about missing packages, install them prior to OpenPivGui:

\begin{sphinxVerbatim}[commandchars=\\\{\}]
\PYG{n}{pip3} \PYG{n}{install} \PYG{n}{missing}\PYG{o}{\PYGZhy{}}\PYG{n}{package}
\end{sphinxVerbatim}

\item[{Something is not working properly.}] \leavevmode
Ensure, you are running the latest version:

\begin{sphinxVerbatim}[commandchars=\\\{\}]
\PYG{n}{pip3} \PYG{n}{install} \PYG{o}{\PYGZhy{}}\PYG{o}{\PYGZhy{}}\PYG{n}{upgrade} \PYG{n}{openpivgui}
\end{sphinxVerbatim}

\item[{Something is still not working properly.}] \leavevmode
Start OpenPivGui from the command line:

\begin{sphinxVerbatim}[commandchars=\\\{\}]
\PYG{n}{python3} \PYG{o}{\PYGZhy{}}\PYG{n}{m} \PYG{n}{openpivgui}\PYG{o}{.}\PYG{n}{OpenPivGui}
\end{sphinxVerbatim}

Check the command line for error messages. Do they provide some useful information?

\item[{I can not see a file list.}] \leavevmode
The GUI may hide some widgets. Toggle to full\sphinxhyphen{}screen mode or try to check the »compact layout« option on the »General« rider.

\item[{I do not understand, how the »back« and »forward« buttons work.}] \leavevmode
All output files are stored in the same directory as the input files. To display a clean list of a single processing step, the content of the working directory can be filtered. The »back« and »forward« buttons change the filter. The filters are defined as a list of comma separated regular expressions in the variable »navigation pattern« on the »General« tab.

Examples:

\sphinxcode{\sphinxupquote{png\$}} Show only files that end on the letters »png«.

\sphinxcode{\sphinxupquote{piv\_{[}0\sphinxhyphen{}9{]}+\textbackslash{}.vec\$}} Show only files that end on \sphinxcode{\sphinxupquote{piv\_}}, followed by a number and \sphinxcode{\sphinxupquote{.vec}}. These are usually the raw results.

\sphinxcode{\sphinxupquote{sig2noise\_repl\textbackslash{}.vec\$}} Final result after applying a validation based on the signal to noise ratio and filling the gaps.

You can learn more about regular expressions by reading the \sphinxhref{https://docs.python.org/3/howto/regex.html\#regex-howto}{Python3 Regular Expression HOWTO}.

\item[{I get »UnidentifiedImageError: cannot identify image file«}] \leavevmode
This happens, when a PIV evaluation is started and the file list contains vector files instead of image files. Press the »back« button until the file list contains image files.

\item[{I get »UnicodeDecodeError: ‘utf\sphinxhyphen{}8’ codec can’t decode byte 0xff in position 85: invalid start byte«}] \leavevmode
This happens, when PIV evaluation is NOT selected and the file list contains image files. Either press the »back button« until the file list contains vector files or select »direct correlation« on the PIV rider.

\end{description}


\chapter{Parameters}
\label{\detokenize{parameters:parameters}}\label{\detokenize{parameters::doc}}

\section{General}
\label{\detokenize{parameters:general}}\begin{description}
\item[{filenames}] \leavevmode
None

\item[{number of cores}] \leavevmode
Select amount of cores to be used for PIV evaluations.

\item[{sequence order step}] \leavevmode
Select sequence order step for evaluation.
Assuming \textgreater{}\textgreater{}skip\textless{}\textless{} = 1; 
\textgreater{}\textgreater{}1\textless{}\textless{} yields (1+2),(2+3)
\textgreater{}\textgreater{}2\textless{}\textless{} yields (1+2),(3+4)

\item[{sequence order skip}] \leavevmode
Select sequence order jump for evaluation.
Assuming \textgreater{}\textgreater{}step\textless{}\textless{} = 1; 
\textgreater{}\textgreater{}1\textless{}\textless{} yields (1+2),(2+3)
\textgreater{}\textgreater{}2\textless{}\textless{} yields (1+3),(2+4)
\textgreater{}\textgreater{}3\textless{}\textless{} yields (1+4),(2+5)
and so on…

\item[{compact layout}] \leavevmode
If selected, the layout is optimized for full screen usage and small screens. Otherwise, the layout leaves some horizontal space for other apps like a terminal window or source code editor. This setting takes effect after restart.

\item[{base output filename}] \leavevmode
Filename for vector output. A number and an acronym that indicates the process history are added automatically.

\item[{navigation pattern}] \leavevmode
Regular expression patterns for filtering the files in the current directory. Use the back and forward buttons to apply a different filter.

\item[{settings for using pandas}] \leavevmode
Individual settings for loading files using pandas.

\item[{skip rows}] \leavevmode
Number of rows skipped at the beginning of the file.

\item[{decimal separator}] \leavevmode
Decimal separator for floating point numbers.

\item[{column separator}] \leavevmode
Column separator.

\item[{read header}] \leavevmode
Read header. If chosen, first line will be interpreted as the header

\item[{specify own header names}] \leavevmode
Specify comma separated list of column names.Example: x,y,vx,vy,sig2noise

\end{description}


\section{Pre\sphinxhyphen{}Processing}
\label{\detokenize{parameters:pre-processing}}\begin{description}
\item[{region of interest}] \leavevmode
Define region of interest.

\item[{x min}] \leavevmode
Defining region of interest.

\item[{x max}] \leavevmode
Defining region of interest.

\item[{y min}] \leavevmode
Defining region of interest.

\item[{y max}] \leavevmode
Defining region of interest.

\item[{invert image}] \leavevmode
Invert image (see skimage invert()).

\item[{Gaussian filter}] \leavevmode
Standard Gaussian blurring filter (see scipy gaussian\_filter()).

\item[{sigma/kernel size}] \leavevmode
Defining the size of the sigma/kernel for gaussian blur filter.

\item[{CLAHE filter}] \leavevmode
Contrast Limited Adaptive Histogram Equalization filter (see skimage adapthist()).

\item[{kernel size}] \leavevmode
Defining the size of the kernel for CLAHE.

\item[{clip limit}] \leavevmode
Defining the contrast with 0\sphinxhyphen{}1 (1 gives highest contrast).

\item[{UnSharp high pass mask/filter}] \leavevmode
A simple image high pass filter (see skimage un\_sharp()).

\item[{perform before CLAHE}] \leavevmode
Perform UnSharp high pass mask/filter before CLAHE.

\item[{filter radius}] \leavevmode
Defining the radius value of the subtracted gaussian filter in the UnSharp high pass mask/filter (positive ints only).

\item[{clip limit}] \leavevmode
Defining the clip of the UnSharp filter (higher values remove more background noise).

\item[{dynamic masking}] \leavevmode
Dynamic masking for masking of images. 
Warning: This is still in testing and is not recommended for use.

\item[{mask type}] \leavevmode
Defining dynamic mask type.

\item[{mask threshold}] \leavevmode
Defining threshold of dynamic mask.

\item[{mask filter size}] \leavevmode
Defining size of the masks.

\end{description}


\section{PIV Evaluation}
\label{\detokenize{parameters:piv-evaluation}}\begin{description}
\item[{do PIV evaluation}] \leavevmode
Do PIV evaluation, select method and parameters below. Deselect, if you just want to do some post\sphinxhyphen{}processing.

\item[{evaluation method}] \leavevmode
extd: Direct correlation with extended size of the search area. 
widim: Window displacement iterative method. (Iterative grid refinement or multi pass PIV). 
windef: Iterative grid refinement with window deformation (recommended).

\item[{search area size}] \leavevmode
Size of square search area in pixel for extd method.

\item[{interrogation window size}] \leavevmode
Size of square interrogation windows in pixel (final pass, in pixel).

\item[{overlap}] \leavevmode
Overlap of correlation windows or vector spacing (final pass, in pixel).

\item[{number of refinement steps}] \leavevmode
Example: A window size of 16 and a number of refinement steps of 2 gives an window size of 64×64 in the fist pass, 32×32 in the second pass and 16×16 pixel in the final pass. (Applies to widim and windef methods only.)

\item[{correlation method}] \leavevmode
Correlation method. Circular is no padding andlinear is zero padding (applies to Windef).

\item[{subpixel method}] \leavevmode
Fit function for determining the subpixel position of the correlation peak.

\item[{signal2noise calculation method}] \leavevmode
Calculation method for the signal to noise ratio.

\item[{dt}] \leavevmode
Interframing time in seconds.

\item[{scale}] \leavevmode
Interframing scaling in pix/m

\item[{invert u\sphinxhyphen{}component}] \leavevmode
Invert (negative) u\sphinxhyphen{}component when saving RAW results.

\item[{invert v\sphinxhyphen{}component}] \leavevmode
Invert (negative) v\sphinxhyphen{}component when saving RAW results.

\end{description}


\section{Validation}
\label{\detokenize{parameters:validation}}\begin{description}
\item[{signal to noise ratio validation}] \leavevmode
Validate the data based on the signal to nose ratio of the cross correlation.

\item[{signal to noise threshold}] \leavevmode
Threshold for filtering based on signal to noise ratio.

\item[{standard deviation validation}] \leavevmode
Validate the data based on a multiple of the standard deviation.

\item[{standard deviation threshold}] \leavevmode
Remove vectors, if the the sum of the squared vector components is larger than the threshold times the standard deviation of the flow field.

\item[{local median validation}] \leavevmode
Validate the data based on a local median threshold.

\item[{local median threshold}] \leavevmode
Discard vector, if the absolute difference with the local median is greater than the threshold.

\item[{global threshold validation}] \leavevmode
Validate the data based on a set global thresholds.

\item[{min u}] \leavevmode
Minimum U allowable component.

\item[{max u}] \leavevmode
Maximum U allowable component.

\item[{min v}] \leavevmode
Minimum V allowable component.

\item[{max v}] \leavevmode
Maximum V allowable component.

\end{description}


\section{Post\sphinxhyphen{}Processing}
\label{\detokenize{parameters:post-processing}}\begin{description}
\item[{replace outliers}] \leavevmode
Replace outliers.

\item[{replacement method}] \leavevmode
Each NaN element is replaced by a weighed averageof neighbours. Localmean uses a square kernel, disk a uniform circular kernel, and distance a kernel with a weight that is proportional to the distance.

\item[{number of iterations}] \leavevmode
If there are adjacent NaN elements, iterative replacement is needed.

\item[{kernel size}] \leavevmode
Diameter of the weighting kernel.

\item[{smoothn data}] \leavevmode
Smoothn data using openpiv.smoothn.

\item[{smoothn vectors}] \leavevmode
Smoothn data with openpiv.smoothn. \textless{}each pass\textgreater{} only applies to windef

\item[{smoothn each pass}] \leavevmode
Smoothn each pass using openpiv.smoothn.

\item[{smoothn robust}] \leavevmode
Activate robust in smoothn (minimizes influence of outlying data).

\item[{smoothning strength}] \leavevmode
Strength of smoothn script. Higher scalar number produces more smoothned data.

\end{description}


\section{Plotting}
\label{\detokenize{parameters:plotting}}\begin{description}
\item[{plot type}] \leavevmode
Select, how to plot velocity data.

\item[{vector scaling}] \leavevmode
Velocity as a fraction of the plot width, e.g.: m/s per plot width. Large values result in shorter vectors.

\item[{vector line width}] \leavevmode
Line width as a fraction of the plot width.

\item[{invalid vector color}] \leavevmode
The color of the invalid vectors

\item[{valid vector color}] \leavevmode
The color of the valid vectors

\item[{vector plot invert y\sphinxhyphen{}axis}] \leavevmode
Define the top left corner as the origin of the vector plot coordinate sytem, as it is common practice in image processing.

\item[{histogram quantity}] \leavevmode
The absolute value of the velocity (v) or its x\sphinxhyphen{} or y\sphinxhyphen{}component (v\_x or v\_y).

\item[{histogram number of bins}] \leavevmode
Number of bins (bars) in the histogram.

\item[{histogram log scale}] \leavevmode
Use a logarithmic y\sphinxhyphen{}axis.

\item[{profiles orientation}] \leavevmode
Plot v\_y over x (horizontal) or v\_x over y (vertical).

\item[{Use pandas plot utility.}] \leavevmode
If chosen, plots will be generated with pandas.

\item[{plot\sphinxhyphen{}type}] \leavevmode
Choose plot\sphinxhyphen{}type. For further information refer to pandas.DataFrame.plot().

\item[{column name for x\sphinxhyphen{}data}] \leavevmode
column name for x\sphinxhyphen{}data. If unknown watch labbook entry.

\item[{column name for y\sphinxhyphen{}data}] \leavevmode
column name for y\sphinxhyphen{}data. If unknown watch labbook entry. For histogram only y\_data are needed.

\item[{number of bins}] \leavevmode
number of bins. This box is only used for plotting type scatter.

\item[{diagram title}] \leavevmode
diagram title.

\item[{grid}] \leavevmode
adds a grid to the diagram.

\item[{legend}] \leavevmode
adds a legend to the diagram.

\item[{axis scaling}] \leavevmode
scales the axes. logarithm scaling x\sphinxhyphen{}axis \textendash{}\textgreater{} logx; logarithm scaling y\sphinxhyphen{}axis \textendash{}\textgreater{} logy; logarithm scaling both axes \textendash{}\textgreater{} loglog.

\item[{limits for the x\sphinxhyphen{}axis}] \leavevmode
For implementation use (lower\_limit, upper\_limit).

\item[{limits for the y\sphinxhyphen{}axis}] \leavevmode
For implementation use (lower\_limit, upper\_limit).

\end{description}


\chapter{Contribution}
\label{\detokenize{contribution:contribution}}\label{\detokenize{contribution::doc}}
Contributions are very welcome! Please have the design considerations in mind and follow the step by step guides below.


\section{Design Considerations}
\label{\detokenize{contribution:design-considerations}}\begin{itemize}
\item {} 
The order of parameters and parameter riders in the GUI should represent the flow of data processing (figure).

\end{itemize}

\begin{figure}[htbp]
\centering
\capstart

\noindent\sphinxincludegraphics{{data_flow}.svg}
\caption{OpenPivGui data processing flow diagram.}\label{\detokenize{contribution:id2}}\end{figure}
\begin{itemize}
\item {} 
OpenPivGui ist designed for providing a very fast workflow. Don’t interrupt this workflow by asking for individual user input, as far as possible. Rather use parameters or algorithms.

\item {} 
Separate the user interface code from the working code. An OpenPivParams object can be created and used independently from the GUI. Also the vec\_plot module or the MultiProcessing class can be called from the command line from another program or Jupyter notebook, independently from the GUI.

\item {} 
The code is designed for developing it together with PIV\sphinxhyphen{}experimentalists rather than programmers. Therefore, keep the code as clear and concise as possible, also by resigning from too many user\sphinxhyphen{}friendly gimmicks.

\item {} 
Write your docstrigs in the \sphinxhref{https://numpydoc.readthedocs.io/en/latest/format.html}{Numpy/Scipy style}.

\end{itemize}


\section{Without Write Access}
\label{\detokenize{contribution:without-write-access}}
This is the standard procedure of contributing to OpenPivGui.
\begin{enumerate}
\sphinxsetlistlabels{\arabic}{enumi}{enumii}{}{.}%
\item {} 
If not done, install Git (platform dependend) and configure it on the command line:

\begin{sphinxVerbatim}[commandchars=\\\{\}]
\PYG{n}{git} \PYG{n}{config} \PYG{o}{\PYGZhy{}}\PYG{o}{\PYGZhy{}}\PYG{k}{global} \PYG{n}{user}\PYG{o}{.}\PYG{n}{name} \PYG{l+s+s2}{\PYGZdq{}}\PYG{l+s+s2}{first name surname}\PYG{l+s+s2}{\PYGZdq{}}
\PYG{n}{git} \PYG{n}{config} \PYG{o}{\PYGZhy{}}\PYG{o}{\PYGZhy{}}\PYG{k}{global} \PYG{n}{user}\PYG{o}{.}\PYG{n}{email} \PYG{l+s+s2}{\PYGZdq{}}\PYG{l+s+s2}{e\PYGZhy{}mail address}\PYG{l+s+s2}{\PYGZdq{}}
\end{sphinxVerbatim}

\item {} 
Create a Github account, navigate to the \sphinxhref{https://github.com/OpenPIV/openpiv\_tk\_gui}{OpenPivGui Github page} and press the fork button (top right of the page). Github will create a personal online fork of the repository for you.

\item {} 
Clone your fork, to get a local copy:

\begin{sphinxVerbatim}[commandchars=\\\{\}]
\PYG{n}{git} \PYG{n}{clone} \PYG{n}{https}\PYG{p}{:}\PYG{o}{/}\PYG{o}{/}\PYG{n}{github}\PYG{o}{.}\PYG{n}{com}\PYG{o}{/}\PYG{n}{your\PYGZus{}user\PYGZus{}name}\PYG{o}{/}\PYG{n}{openpiv\PYGZus{}tk\PYGZus{}gui}\PYG{o}{.}\PYG{n}{git}
\end{sphinxVerbatim}

\item {} 
Your fork is independent from the original (upstream) repository. To be able to sync changes in the upstream repository with your fork later, specify the upstream repository:

\begin{sphinxVerbatim}[commandchars=\\\{\}]
\PYG{n}{cd} \PYG{n}{openpiv\PYGZus{}tk\PYGZus{}gui}
\PYG{n}{git} \PYG{n}{remote} \PYG{n}{add} \PYG{n}{upstream} \PYG{n}{https}\PYG{p}{:}\PYG{o}{/}\PYG{o}{/}\PYG{n}{github}\PYG{o}{.}\PYG{n}{com}\PYG{o}{/}\PYG{n}{OpenPIV}\PYG{o}{/}\PYG{n}{openpiv\PYGZus{}tk\PYGZus{}gui}\PYG{o}{.}\PYG{n}{git}
\PYG{n}{git} \PYG{n}{remote} \PYG{o}{\PYGZhy{}}\PYG{n}{v}
\end{sphinxVerbatim}

\item {} 
Change the code locally, commit the changes:

\begin{sphinxVerbatim}[commandchars=\\\{\}]
\PYG{n}{git} \PYG{n}{add} \PYG{o}{.}
\PYG{n}{git} \PYG{n}{commit} \PYG{o}{\PYGZhy{}}\PYG{n}{m} \PYG{l+s+s1}{\PYGZsq{}}\PYG{l+s+s1}{A meaningful comment on the changes.}\PYG{l+s+s1}{\PYGZsq{}}
\end{sphinxVerbatim}

\item {} 
See, if there are updates in the upstream repository and save them in your local branch upstream/master::

\begin{sphinxVerbatim}[commandchars=\\\{\}]
\PYG{n}{git} \PYG{n}{fetch} \PYG{n}{upstream}
\end{sphinxVerbatim}

\item {} 
Merge possible upstream changes into your local master branch:

\begin{sphinxVerbatim}[commandchars=\\\{\}]
\PYG{n}{git} \PYG{n}{merge} \PYG{n}{upstream}\PYG{o}{/}\PYG{n}{master}
\end{sphinxVerbatim}

\item {} 
If there are merge conflicts, use \sphinxcode{\sphinxupquote{git status}} and \sphinxcode{\sphinxupquote{git diff}} for displaying them. Git marks conflicts in your files, \sphinxhref{https://docs.github.com/en/github/collaborating-with-issues-and-pull-requests/resolving-a-merge-conflict-using-the-command-line}{as described in the Github documentation on solving merge conflicts}. After resolving merge conflicts, upload everything:

\begin{sphinxVerbatim}[commandchars=\\\{\}]
\PYG{n}{git} \PYG{n}{add} \PYG{o}{.}
\PYG{n}{git} \PYG{n}{commit} \PYG{o}{\PYGZhy{}}\PYG{n}{m} \PYG{l+s+s1}{\PYGZsq{}}\PYG{l+s+s1}{A meaningful comment.}\PYG{l+s+s1}{\PYGZsq{}}
\PYG{n}{git} \PYG{n}{push}
\end{sphinxVerbatim}

\item {} 
Propose your changes to the upstream developer by creating a pull\sphinxhyphen{}request, as described \sphinxhref{https://docs.github.com/en/github/collaborating-with-issues-and-pull-requests/creating-a-pull-request-from-a-fork}{in the Github documentation for creating a pull\sphinxhyphen{}request from a fork}. (Basically just pressing the »New pull request« button.)

\end{enumerate}

Good luck!


\section{With Write Access}
\label{\detokenize{contribution:with-write-access}}\begin{enumerate}
\sphinxsetlistlabels{\arabic}{enumi}{enumii}{}{.}%
\item {} 
If not done, install Git and configure it:

\begin{sphinxVerbatim}[commandchars=\\\{\}]
\PYG{n}{git} \PYG{n}{config} \PYG{o}{\PYGZhy{}}\PYG{o}{\PYGZhy{}}\PYG{k}{global} \PYG{n}{user}\PYG{o}{.}\PYG{n}{name} \PYG{l+s+s2}{\PYGZdq{}}\PYG{l+s+s2}{first name surname}\PYG{l+s+s2}{\PYGZdq{}}
\PYG{n}{git} \PYG{n}{config} \PYG{o}{\PYGZhy{}}\PYG{o}{\PYGZhy{}}\PYG{k}{global} \PYG{n}{user}\PYG{o}{.}\PYG{n}{email} \PYG{l+s+s2}{\PYGZdq{}}\PYG{l+s+s2}{e\PYGZhy{}mail address}\PYG{l+s+s2}{\PYGZdq{}}
\end{sphinxVerbatim}

\item {} 
Clone the git repository:

\begin{sphinxVerbatim}[commandchars=\\\{\}]
\PYG{n}{git} \PYG{n}{clone} \PYG{n}{https}\PYG{p}{:}\PYG{o}{/}\PYG{o}{/}\PYG{n}{github}\PYG{o}{.}\PYG{n}{com}\PYG{o}{/}\PYG{n}{OpenPIV}\PYG{o}{/}\PYG{n}{openpiv\PYGZus{}tk\PYGZus{}gui}\PYG{o}{.}\PYG{n}{git}
\end{sphinxVerbatim}

\item {} 
Create a new branch and switch over to it:

\begin{sphinxVerbatim}[commandchars=\\\{\}]
\PYG{n}{cd} \PYG{n}{openpiv\PYGZus{}tk\PYGZus{}gui}
\PYG{n}{git} \PYG{n}{branch} \PYG{n}{meaningful}\PYG{o}{\PYGZhy{}}\PYG{n}{branch}\PYG{o}{\PYGZhy{}}\PYG{n}{name}
\PYG{n}{git} \PYG{n}{checkout} \PYG{n}{meaningful}\PYG{o}{\PYGZhy{}}\PYG{n}{branch}\PYG{o}{\PYGZhy{}}\PYG{n}{name}
\PYG{n}{git} \PYG{n}{status}
\end{sphinxVerbatim}

\item {} 
Change the code locally and commit changes:

\begin{sphinxVerbatim}[commandchars=\\\{\}]
\PYG{n}{git} \PYG{n}{add} \PYG{o}{.}
\PYG{n}{git} \PYG{n}{commit} \PYG{o}{\PYGZhy{}}\PYG{n}{m} \PYG{l+s+s1}{\PYGZsq{}}\PYG{l+s+s1}{A meaningful comment on the changes.}\PYG{l+s+s1}{\PYGZsq{}}
\end{sphinxVerbatim}

\item {} 
Push branch, so everyone can see it:

\begin{sphinxVerbatim}[commandchars=\\\{\}]
\PYG{n}{git} \PYG{n}{push} \PYG{o}{\PYGZhy{}}\PYG{o}{\PYGZhy{}}\PYG{n+nb}{set}\PYG{o}{\PYGZhy{}}\PYG{n}{upstream} \PYG{n}{origin} \PYG{n}{meaningful}\PYG{o}{\PYGZhy{}}\PYG{n}{branch}\PYG{o}{\PYGZhy{}}\PYG{n}{name}
\end{sphinxVerbatim}

\item {} 
Create a pull request. This is not a Git, but a Github feature, so you must use the Github user\sphinxhyphen{}interface, as described in the \sphinxhref{https://docs.github.com/en/github/collaborating-with-issues-and-pull-requests/creating-a-pull-request\#creating-the-pull-request}{Github documentaton on creating a pull request from a branch}.

\item {} 
After discussing the changes and possibly additional commits, the feature\sphinxhyphen{}branch can be merged into the main branch:

\begin{sphinxVerbatim}[commandchars=\\\{\}]
\PYG{n}{git} \PYG{n}{checkout} \PYG{n}{master}
\PYG{n}{git} \PYG{n}{merge} \PYG{n}{meaningful}\PYG{o}{\PYGZhy{}}\PYG{n}{branch}\PYG{o}{\PYGZhy{}}\PYG{n}{name}
\end{sphinxVerbatim}

\item {} 
Eventually, solve merge conflicts. Use \sphinxcode{\sphinxupquote{git status}} and \sphinxcode{\sphinxupquote{git diff}} for displaying conflicts. Git marks conflicts in your files, \sphinxhref{https://docs.github.com/en/github/collaborating-with-issues-and-pull-requests/resolving-a-merge-conflict-using-the-command-line}{as described in the Github documentation on solving merge conflicts}.

\item {} 
Finally, the feature\sphinxhyphen{}branch can safely be removed:

\begin{sphinxVerbatim}[commandchars=\\\{\}]
\PYG{n}{git} \PYG{n}{branch} \PYG{o}{\PYGZhy{}}\PYG{n}{d} \PYG{n}{meaningful}\PYG{o}{\PYGZhy{}}\PYG{n}{branch}\PYG{o}{\PYGZhy{}}\PYG{n}{name}
\end{sphinxVerbatim}

\item {} 
Go to the Github user\sphinxhyphen{}interface and also delete the now obsolete online copy of the feature\sphinxhyphen{}branch.

\end{enumerate}

Good luck!


\chapter{Code Documentation}
\label{\detokenize{code_doc:code-documentation}}\label{\detokenize{code_doc::doc}}
The code is hosted on Github: \sphinxurl{https://github.com/OpenPIV/openpiv\_tk\_gui}.


\section{OpenPivGui}
\label{\detokenize{openpivgui:module-openpivgui.OpenPivGui}}\label{\detokenize{openpivgui:openpivgui}}\label{\detokenize{openpivgui::doc}}\index{module@\spxentry{module}!openpivgui.OpenPivGui@\spxentry{openpivgui.OpenPivGui}}\index{openpivgui.OpenPivGui@\spxentry{openpivgui.OpenPivGui}!module@\spxentry{module}}
A simple GUI for OpenPIV.
\index{OpenPivGui (class in openpivgui.OpenPivGui)@\spxentry{OpenPivGui}\spxextra{class in openpivgui.OpenPivGui}}

\begin{fulllineitems}
\phantomsection\label{\detokenize{openpivgui:openpivgui.OpenPivGui.OpenPivGui}}\pysigline{\sphinxbfcode{\sphinxupquote{class }}\sphinxcode{\sphinxupquote{openpivgui.OpenPivGui.}}\sphinxbfcode{\sphinxupquote{OpenPivGui}}}
Simple OpenPIV GUI

Usage:

1. Press »select files« and choose some images.
Use Ctrl + Shift for selecting mutliple files.
\begin{enumerate}
\sphinxsetlistlabels{\arabic}{enumi}{enumii}{}{.}%
\setcounter{enumi}{1}
\item {} 
Click on the links in the file\sphinxhyphen{}list to inspect the images.

\end{enumerate}

3. Walk through the riders, select the desired functions,
and edit the corresponding parameters.
\begin{enumerate}
\sphinxsetlistlabels{\arabic}{enumi}{enumii}{}{.}%
\setcounter{enumi}{3}
\item {} 
Press »start processing« to start the processing.

\item {} 
Inspect the results by clicking on the links in the file\sphinxhyphen{}list.

\end{enumerate}

6. Use the »back« and »forward« buttons to inspect
intermediate results.

4. Use »dump settings« to document your project. You can recall them
anytime by pressing »load settings«. The lab\sphinxhyphen{}book entries
are also restored from the settings file.

See also:

\sphinxurl{https://github.com/OpenPIV/openpiv\_tk\_gui}
\index{delete\_files() (openpivgui.OpenPivGui.OpenPivGui method)@\spxentry{delete\_files()}\spxextra{openpivgui.OpenPivGui.OpenPivGui method}}

\begin{fulllineitems}
\phantomsection\label{\detokenize{openpivgui:openpivgui.OpenPivGui.OpenPivGui.delete_files}}\pysiglinewithargsret{\sphinxbfcode{\sphinxupquote{delete\_files}}}{}{}
Delete files currently listed in the file list.

\end{fulllineitems}

\index{destroy() (openpivgui.OpenPivGui.OpenPivGui method)@\spxentry{destroy()}\spxextra{openpivgui.OpenPivGui.OpenPivGui method}}

\begin{fulllineitems}
\phantomsection\label{\detokenize{openpivgui:openpivgui.OpenPivGui.OpenPivGui.destroy}}\pysiglinewithargsret{\sphinxbfcode{\sphinxupquote{destroy}}}{}{}
Destroy the OpenPIV GUI.

Settings are automatically saved.

\end{fulllineitems}

\index{file\_filter() (openpivgui.OpenPivGui.OpenPivGui method)@\spxentry{file\_filter()}\spxextra{openpivgui.OpenPivGui.OpenPivGui method}}

\begin{fulllineitems}
\phantomsection\label{\detokenize{openpivgui:openpivgui.OpenPivGui.OpenPivGui.file_filter}}\pysiglinewithargsret{\sphinxbfcode{\sphinxupquote{file\_filter}}}{\emph{\DUrole{n}{files}}, \emph{\DUrole{n}{pattern}}}{}
Filter a list of files to  match a pattern.
\begin{quote}\begin{description}
\item[{Parameters}] \leavevmode\begin{itemize}
\item {} 
\sphinxstyleliteralstrong{\sphinxupquote{files}} (\sphinxstyleliteralemphasis{\sphinxupquote{str}}\sphinxstyleliteralemphasis{\sphinxupquote{{[}}}\sphinxstyleliteralemphasis{\sphinxupquote{{]}}}) \textendash{} A list of pathnames.

\item {} 
\sphinxstyleliteralstrong{\sphinxupquote{pattern}} (\sphinxstyleliteralemphasis{\sphinxupquote{str}}) \textendash{} A regular expression for filtering the list.

\end{itemize}

\item[{Returns}] \leavevmode
List items that match the pattern.

\item[{Return type}] \leavevmode
str{[}{]}

\end{description}\end{quote}

\end{fulllineitems}

\index{get\_filelistbox() (openpivgui.OpenPivGui.OpenPivGui method)@\spxentry{get\_filelistbox()}\spxextra{openpivgui.OpenPivGui.OpenPivGui method}}

\begin{fulllineitems}
\phantomsection\label{\detokenize{openpivgui:openpivgui.OpenPivGui.OpenPivGui.get_filelistbox}}\pysiglinewithargsret{\sphinxbfcode{\sphinxupquote{get\_filelistbox}}}{}{}
Return a handle to the file list widget.
\begin{quote}\begin{description}
\item[{Returns}] \leavevmode
A handle to the listbox widget holding the filenames

\item[{Return type}] \leavevmode
tkinter.Listbox

\end{description}\end{quote}

\end{fulllineitems}

\index{get\_settings() (openpivgui.OpenPivGui.OpenPivGui method)@\spxentry{get\_settings()}\spxextra{openpivgui.OpenPivGui.OpenPivGui method}}

\begin{fulllineitems}
\phantomsection\label{\detokenize{openpivgui:openpivgui.OpenPivGui.OpenPivGui.get_settings}}\pysiglinewithargsret{\sphinxbfcode{\sphinxupquote{get\_settings}}}{}{}
Copy widget variables to the parameter object.

\end{fulllineitems}

\index{load\_pandas() (openpivgui.OpenPivGui.OpenPivGui method)@\spxentry{load\_pandas()}\spxextra{openpivgui.OpenPivGui.OpenPivGui method}}

\begin{fulllineitems}
\phantomsection\label{\detokenize{openpivgui:openpivgui.OpenPivGui.OpenPivGui.load_pandas}}\pysiglinewithargsret{\sphinxbfcode{\sphinxupquote{load\_pandas}}}{\emph{\DUrole{n}{fname}}}{}
Load files in a pandas data frame.

On the rider named General, the parameters for loading
the data frames can be specified.
No parameters have to be set for image processing.
\begin{quote}\begin{description}
\item[{Parameters}] \leavevmode
\sphinxstyleliteralstrong{\sphinxupquote{fname}} \textendash{} A filename.

\item[{Returns}] \leavevmode
In case of an error, the errormessage is returned (str).

\item[{Return type}] \leavevmode
pandas.DataFrame

\end{description}\end{quote}

\end{fulllineitems}

\index{load\_settings() (openpivgui.OpenPivGui.OpenPivGui method)@\spxentry{load\_settings()}\spxextra{openpivgui.OpenPivGui.OpenPivGui method}}

\begin{fulllineitems}
\phantomsection\label{\detokenize{openpivgui:openpivgui.OpenPivGui.OpenPivGui.load_settings}}\pysiglinewithargsret{\sphinxbfcode{\sphinxupquote{load\_settings}}}{}{}
Load settings from a JSON file.

\end{fulllineitems}

\index{log() (openpivgui.OpenPivGui.OpenPivGui method)@\spxentry{log()}\spxextra{openpivgui.OpenPivGui.OpenPivGui method}}

\begin{fulllineitems}
\phantomsection\label{\detokenize{openpivgui:openpivgui.OpenPivGui.OpenPivGui.log}}\pysiglinewithargsret{\sphinxbfcode{\sphinxupquote{log}}}{\emph{\DUrole{n}{columninformation}\DUrole{o}{=}\DUrole{default_value}{None}}, \emph{\DUrole{n}{timestamp}\DUrole{o}{=}\DUrole{default_value}{False}}, \emph{\DUrole{n}{text}\DUrole{o}{=}\DUrole{default_value}{None}}, \emph{\DUrole{n}{group}\DUrole{o}{=}\DUrole{default_value}{None}}}{}
Add an entry to the lab\sphinxhyphen{}book.

The first initialized text\sphinxhyphen{}area is assumed to be the lab\sphinxhyphen{}book.
It is internally accessible by self.ta{[}0{]}.
\begin{quote}\begin{description}
\item[{Parameters}] \leavevmode\begin{itemize}
\item {} 
\sphinxstyleliteralstrong{\sphinxupquote{timestamp}} (\sphinxstyleliteralemphasis{\sphinxupquote{bool}}) \textendash{} Print current time.
Pattern: yyyy\sphinxhyphen{}mm\sphinxhyphen{}dd hh:mm:ss.
(default: False)

\item {} 
\sphinxstyleliteralstrong{\sphinxupquote{text}} (\sphinxstyleliteralemphasis{\sphinxupquote{str}}) \textendash{} Print a text, a linebreak is appended.
(default None)

\item {} 
\sphinxstyleliteralstrong{\sphinxupquote{group}} (\sphinxstyleliteralemphasis{\sphinxupquote{int}}) \textendash{} Print group of parameters.
(e.g. OpenPivParams.PIVPROC)

\item {} 
\sphinxstyleliteralstrong{\sphinxupquote{columninformation}} (\sphinxstyleliteralemphasis{\sphinxupquote{list}}) \textendash{} Print column information of the selected file.

\end{itemize}

\end{description}\end{quote}
\subsubsection*{Example}
\begin{description}
\item[{log(text=’processing parameters:’,}] \leavevmode
group=OpenPivParams.POSTPROC)

\end{description}

\end{fulllineitems}

\index{move\_files() (openpivgui.OpenPivGui.OpenPivGui method)@\spxentry{move\_files()}\spxextra{openpivgui.OpenPivGui.OpenPivGui method}}

\begin{fulllineitems}
\phantomsection\label{\detokenize{openpivgui:openpivgui.OpenPivGui.OpenPivGui.move_files}}\pysiglinewithargsret{\sphinxbfcode{\sphinxupquote{move\_files}}}{}{}
Move files to a new place.

\end{fulllineitems}

\index{navigate() (openpivgui.OpenPivGui.OpenPivGui method)@\spxentry{navigate()}\spxextra{openpivgui.OpenPivGui.OpenPivGui method}}

\begin{fulllineitems}
\phantomsection\label{\detokenize{openpivgui:openpivgui.OpenPivGui.OpenPivGui.navigate}}\pysiglinewithargsret{\sphinxbfcode{\sphinxupquote{navigate}}}{\emph{\DUrole{n}{direction}}}{}
Navigate through processing steps.

Display a filtered list of files of the current
directory. This function cycles through the filters
specified by the key ‘navi\_pattern’ in the settings object.
\begin{quote}\begin{description}
\item[{Parameters}] \leavevmode
\sphinxstyleliteralstrong{\sphinxupquote{direction}} (\sphinxstyleliteralemphasis{\sphinxupquote{str}}) \textendash{} ‘back’ or ‘forward’.

\end{description}\end{quote}

\end{fulllineitems}

\index{open\_directory() (openpivgui.OpenPivGui.OpenPivGui method)@\spxentry{open\_directory()}\spxextra{openpivgui.OpenPivGui.OpenPivGui method}}

\begin{fulllineitems}
\phantomsection\label{\detokenize{openpivgui:openpivgui.OpenPivGui.OpenPivGui.open_directory}}\pysiglinewithargsret{\sphinxbfcode{\sphinxupquote{open\_directory}}}{}{}
Show a dialog for opening a directory.

\end{fulllineitems}

\index{processing() (openpivgui.OpenPivGui.OpenPivGui method)@\spxentry{processing()}\spxextra{openpivgui.OpenPivGui.OpenPivGui method}}

\begin{fulllineitems}
\phantomsection\label{\detokenize{openpivgui:openpivgui.OpenPivGui.OpenPivGui.processing}}\pysiglinewithargsret{\sphinxbfcode{\sphinxupquote{processing}}}{}{}
Start the processing chain.

This is the place to implement additional function calls.

\end{fulllineitems}

\index{readme() (openpivgui.OpenPivGui.OpenPivGui method)@\spxentry{readme()}\spxextra{openpivgui.OpenPivGui.OpenPivGui method}}

\begin{fulllineitems}
\phantomsection\label{\detokenize{openpivgui:openpivgui.OpenPivGui.OpenPivGui.readme}}\pysiglinewithargsret{\sphinxbfcode{\sphinxupquote{readme}}}{}{}
Opens \sphinxurl{https://github.com/OpenPIV/openpiv\_tk\_gui}.

\end{fulllineitems}

\index{reset\_params() (openpivgui.OpenPivGui.OpenPivGui method)@\spxentry{reset\_params()}\spxextra{openpivgui.OpenPivGui.OpenPivGui method}}

\begin{fulllineitems}
\phantomsection\label{\detokenize{openpivgui:openpivgui.OpenPivGui.OpenPivGui.reset_params}}\pysiglinewithargsret{\sphinxbfcode{\sphinxupquote{reset\_params}}}{}{}
Reset parameters to default values.

\end{fulllineitems}

\index{select\_image\_files() (openpivgui.OpenPivGui.OpenPivGui method)@\spxentry{select\_image\_files()}\spxextra{openpivgui.OpenPivGui.OpenPivGui method}}

\begin{fulllineitems}
\phantomsection\label{\detokenize{openpivgui:openpivgui.OpenPivGui.OpenPivGui.select_image_files}}\pysiglinewithargsret{\sphinxbfcode{\sphinxupquote{select\_image\_files}}}{}{}
Show a file dialog to select one or more filenames.

\end{fulllineitems}

\index{set\_settings() (openpivgui.OpenPivGui.OpenPivGui method)@\spxentry{set\_settings()}\spxextra{openpivgui.OpenPivGui.OpenPivGui method}}

\begin{fulllineitems}
\phantomsection\label{\detokenize{openpivgui:openpivgui.OpenPivGui.OpenPivGui.set_settings}}\pysiglinewithargsret{\sphinxbfcode{\sphinxupquote{set\_settings}}}{}{}
Copy values of the parameter object to widget variables.

\end{fulllineitems}

\index{show() (openpivgui.OpenPivGui.OpenPivGui method)@\spxentry{show()}\spxextra{openpivgui.OpenPivGui.OpenPivGui method}}

\begin{fulllineitems}
\phantomsection\label{\detokenize{openpivgui:openpivgui.OpenPivGui.OpenPivGui.show}}\pysiglinewithargsret{\sphinxbfcode{\sphinxupquote{show}}}{\emph{\DUrole{n}{fname}}}{}
Display a file.

This method distinguishes vector data (file extensions
txt, dat, jvc,vec and csv) and images (all other file extensions).
\begin{quote}\begin{description}
\item[{Parameters}] \leavevmode
\sphinxstyleliteralstrong{\sphinxupquote{fname}} (\sphinxstyleliteralemphasis{\sphinxupquote{str}}) \textendash{} A filename.

\end{description}\end{quote}

\end{fulllineitems}

\index{show\_img() (openpivgui.OpenPivGui.OpenPivGui method)@\spxentry{show\_img()}\spxextra{openpivgui.OpenPivGui.OpenPivGui method}}

\begin{fulllineitems}
\phantomsection\label{\detokenize{openpivgui:openpivgui.OpenPivGui.OpenPivGui.show_img}}\pysiglinewithargsret{\sphinxbfcode{\sphinxupquote{show\_img}}}{\emph{\DUrole{n}{fname}}}{}
Display an image.
\begin{quote}\begin{description}
\item[{Parameters}] \leavevmode
\sphinxstyleliteralstrong{\sphinxupquote{fname}} (\sphinxstyleliteralemphasis{\sphinxupquote{str}}) \textendash{} Pathname of an image file.

\end{description}\end{quote}

\end{fulllineitems}

\index{show\_informations() (openpivgui.OpenPivGui.OpenPivGui method)@\spxentry{show\_informations()}\spxextra{openpivgui.OpenPivGui.OpenPivGui method}}

\begin{fulllineitems}
\phantomsection\label{\detokenize{openpivgui:openpivgui.OpenPivGui.OpenPivGui.show_informations}}\pysiglinewithargsret{\sphinxbfcode{\sphinxupquote{show\_informations}}}{\emph{\DUrole{n}{fname}}}{}
Shows the column names of the chosen file in the labbook.
\begin{quote}\begin{description}
\item[{Parameters}] \leavevmode
\sphinxstyleliteralstrong{\sphinxupquote{fname}} (\sphinxstyleliteralemphasis{\sphinxupquote{str}}) \textendash{} A filename.

\end{description}\end{quote}

\end{fulllineitems}

\index{start\_processing() (openpivgui.OpenPivGui.OpenPivGui method)@\spxentry{start\_processing()}\spxextra{openpivgui.OpenPivGui.OpenPivGui method}}

\begin{fulllineitems}
\phantomsection\label{\detokenize{openpivgui:openpivgui.OpenPivGui.OpenPivGui.start_processing}}\pysiglinewithargsret{\sphinxbfcode{\sphinxupquote{start\_processing}}}{}{}
Wrapper function to start processing in a separate thread.

\end{fulllineitems}

\index{user\_function() (openpivgui.OpenPivGui.OpenPivGui method)@\spxentry{user\_function()}\spxextra{openpivgui.OpenPivGui.OpenPivGui method}}

\begin{fulllineitems}
\phantomsection\label{\detokenize{openpivgui:openpivgui.OpenPivGui.OpenPivGui.user_function}}\pysiglinewithargsret{\sphinxbfcode{\sphinxupquote{user\_function}}}{}{}
Executes user function.

\end{fulllineitems}


\end{fulllineitems}



\section{OpenPivParams}
\label{\detokenize{openpivparams:module-openpivgui.OpenPivParams}}\label{\detokenize{openpivparams:openpivparams}}\label{\detokenize{openpivparams::doc}}\index{module@\spxentry{module}!openpivgui.OpenPivParams@\spxentry{openpivgui.OpenPivParams}}\index{openpivgui.OpenPivParams@\spxentry{openpivgui.OpenPivParams}!module@\spxentry{module}}
A class for simple parameter handling.

This class is also used as a basis for automated widget creation
by OpenPivGui.
\index{OpenPivParams (class in openpivgui.OpenPivParams)@\spxentry{OpenPivParams}\spxextra{class in openpivgui.OpenPivParams}}

\begin{fulllineitems}
\phantomsection\label{\detokenize{openpivparams:openpivgui.OpenPivParams.OpenPivParams}}\pysigline{\sphinxbfcode{\sphinxupquote{class }}\sphinxcode{\sphinxupquote{openpivgui.OpenPivParams.}}\sphinxbfcode{\sphinxupquote{OpenPivParams}}}
A class for convenient parameter handling.

Widgets are automatically created based on the content of the
variables in the dictionary OpenPivParams.default.

The entries in OpenPivParams.default are assumed to follow this
pattern:
\begin{description}
\item[{(str) key:}] \leavevmode\begin{description}
\item[{{[}(int) index,}] \leavevmode
(str) type,
value,
(tuple) hints,
(str) label,
(str) help{]}

\end{description}

\end{description}

The index is used for sorting and grouping, because Python
dictionaries below version 3.7 do not preserve their order. A
corresponding input widged ist chosen based on the type string:
\begin{quote}

None:                    no widget, no variable, but a rider
boolean:                 checkbox
str{[}{]}:                   listbox
text:                    text area
other (float, int, str): entry (if hints not None: option menu)
\end{quote}

A label is placed next to each input widget. The help string is
displayed as a tooltip.

The parameter value is directly accessible via indexing the base
variable name. For example, if your OpenPivParams object variable
name is »my\_settings«, you can access a value by typing:

my\_settings{[}key{]}

This is a shortcut for my\_settings.param{[}key{]}. To access other
fields, use my\_settings.label{[}key{]}, my\_settings.help{[}key{]} and so on.
\index{dump\_settings() (openpivgui.OpenPivParams.OpenPivParams method)@\spxentry{dump\_settings()}\spxextra{openpivgui.OpenPivParams.OpenPivParams method}}

\begin{fulllineitems}
\phantomsection\label{\detokenize{openpivparams:openpivgui.OpenPivParams.OpenPivParams.dump_settings}}\pysiglinewithargsret{\sphinxbfcode{\sphinxupquote{dump\_settings}}}{\emph{\DUrole{n}{fname}}}{}
Dump parameter values to a JSON file.
\begin{quote}\begin{description}
\item[{Parameters}] \leavevmode
\sphinxstyleliteralstrong{\sphinxupquote{fname}} (\sphinxstyleliteralemphasis{\sphinxupquote{str}}) \textendash{} A filename.

\end{description}\end{quote}

Only the parameter values are saved. Other data like
index, hint, label and help should only be defined in the
default dictionary in this source code.

\end{fulllineitems}

\index{generate\_parameter\_documentation() (openpivgui.OpenPivParams.OpenPivParams method)@\spxentry{generate\_parameter\_documentation()}\spxextra{openpivgui.OpenPivParams.OpenPivParams method}}

\begin{fulllineitems}
\phantomsection\label{\detokenize{openpivparams:openpivgui.OpenPivParams.OpenPivParams.generate_parameter_documentation}}\pysiglinewithargsret{\sphinxbfcode{\sphinxupquote{generate\_parameter\_documentation}}}{\emph{\DUrole{n}{group}\DUrole{o}{=}\DUrole{default_value}{None}}}{}
Return parameter labels and help as reStructuredText def list.
\begin{quote}\begin{description}
\item[{Parameters}] \leavevmode
\sphinxstyleliteralstrong{\sphinxupquote{group}} (\sphinxstyleliteralemphasis{\sphinxupquote{int}}) \textendash{} Parameter group.
(e.g. OpenPivParams.PIVPROC)

\item[{Returns}] \leavevmode
\sphinxstylestrong{str}

\item[{Return type}] \leavevmode
A reStructuredText definition list for documentation.

\end{description}\end{quote}

\end{fulllineitems}

\index{load\_settings() (openpivgui.OpenPivParams.OpenPivParams method)@\spxentry{load\_settings()}\spxextra{openpivgui.OpenPivParams.OpenPivParams method}}

\begin{fulllineitems}
\phantomsection\label{\detokenize{openpivparams:openpivgui.OpenPivParams.OpenPivParams.load_settings}}\pysiglinewithargsret{\sphinxbfcode{\sphinxupquote{load\_settings}}}{\emph{\DUrole{n}{fname}}}{}
Read parameters from a JSON file.
\begin{quote}\begin{description}
\item[{Parameters}] \leavevmode
\sphinxstyleliteralstrong{\sphinxupquote{fname}} (\sphinxstyleliteralemphasis{\sphinxupquote{str}}) \textendash{} Path of the settings file in JSON format.

\end{description}\end{quote}

Reads only parameter values. Content of the fields index,
type, hint, label and help are always read from the default
dictionary. The default dictionary may contain more entries
than the JSON file (ensuring backwards compatibility).

\end{fulllineitems}


\end{fulllineitems}



\section{MultiProcessing}
\label{\detokenize{multiprocessing:module-openpivgui.MultiProcessing}}\label{\detokenize{multiprocessing:multiprocessing}}\label{\detokenize{multiprocessing::doc}}\index{module@\spxentry{module}!openpivgui.MultiProcessing@\spxentry{openpivgui.MultiProcessing}}\index{openpivgui.MultiProcessing@\spxentry{openpivgui.MultiProcessing}!module@\spxentry{module}}
Parallel Processing of PIV images.
\index{MultiProcessing (class in openpivgui.MultiProcessing)@\spxentry{MultiProcessing}\spxextra{class in openpivgui.MultiProcessing}}

\begin{fulllineitems}
\phantomsection\label{\detokenize{multiprocessing:openpivgui.MultiProcessing.MultiProcessing}}\pysiglinewithargsret{\sphinxbfcode{\sphinxupquote{class }}\sphinxcode{\sphinxupquote{openpivgui.MultiProcessing.}}\sphinxbfcode{\sphinxupquote{MultiProcessing}}}{\emph{\DUrole{n}{params}}}{}
Parallel processing, based on the corrresponding OpenPIV class.

Do not run from the interactive shell or within IDLE! Details at:
\sphinxurl{https://docs.python.org/3.6/library/multiprocessing.html\#using-a-pool-of-workers}
\begin{quote}\begin{description}
\item[{Parameters}] \leavevmode
\sphinxstyleliteralstrong{\sphinxupquote{params}} ({\hyperref[\detokenize{openpivparams:openpivgui.OpenPivParams.OpenPivParams}]{\sphinxcrossref{\sphinxstyleliteralemphasis{\sphinxupquote{OpenPivParams}}}}}) \textendash{} A parameter object.

\end{description}\end{quote}
\index{get\_save\_fnames() (openpivgui.MultiProcessing.MultiProcessing method)@\spxentry{get\_save\_fnames()}\spxextra{openpivgui.MultiProcessing.MultiProcessing method}}

\begin{fulllineitems}
\phantomsection\label{\detokenize{multiprocessing:openpivgui.MultiProcessing.MultiProcessing.get_save_fnames}}\pysiglinewithargsret{\sphinxbfcode{\sphinxupquote{get\_save\_fnames}}}{}{}
Return a list of result filenames.
\begin{quote}\begin{description}
\item[{Returns}] \leavevmode
List of filenames with resulting PIV data.

\item[{Return type}] \leavevmode
str{[}{]}

\end{description}\end{quote}

\end{fulllineitems}

\index{process() (openpivgui.MultiProcessing.MultiProcessing method)@\spxentry{process()}\spxextra{openpivgui.MultiProcessing.MultiProcessing method}}

\begin{fulllineitems}
\phantomsection\label{\detokenize{multiprocessing:openpivgui.MultiProcessing.MultiProcessing.process}}\pysiglinewithargsret{\sphinxbfcode{\sphinxupquote{process}}}{\emph{\DUrole{n}{args}}}{}
Process chain as configured in the GUI.
\begin{quote}\begin{description}
\item[{Parameters}] \leavevmode
\sphinxstyleliteralstrong{\sphinxupquote{args}} (\sphinxstyleliteralemphasis{\sphinxupquote{tuple}}) \textendash{} Tuple as expected by the inherited run method:
file\_a (str) \textendash{} image file a
file\_b (str) \textendash{} image file b
counter (int) \textendash{} index pointing to an element of the filename list

\end{description}\end{quote}

\end{fulllineitems}


\end{fulllineitems}



\section{Postprocessing}
\label{\detokenize{postprocessing:module-openpivgui.PostProcessing}}\label{\detokenize{postprocessing:postprocessing}}\label{\detokenize{postprocessing::doc}}\index{module@\spxentry{module}!openpivgui.PostProcessing@\spxentry{openpivgui.PostProcessing}}\index{openpivgui.PostProcessing@\spxentry{openpivgui.PostProcessing}!module@\spxentry{module}}
Post Processing for OpenPIVGui.
\index{PostProcessing (class in openpivgui.PostProcessing)@\spxentry{PostProcessing}\spxextra{class in openpivgui.PostProcessing}}

\begin{fulllineitems}
\phantomsection\label{\detokenize{postprocessing:openpivgui.PostProcessing.PostProcessing}}\pysiglinewithargsret{\sphinxbfcode{\sphinxupquote{class }}\sphinxcode{\sphinxupquote{openpivgui.PostProcessing.}}\sphinxbfcode{\sphinxupquote{PostProcessing}}}{\emph{\DUrole{n}{params}}}{}
Post Processing routines for vector data.
\begin{quote}\begin{description}
\item[{Parameters}] \leavevmode
\sphinxstyleliteralstrong{\sphinxupquote{params}} (\sphinxstyleliteralemphasis{\sphinxupquote{openpivgui.OpenPivParams}}) \textendash{} Parameter object.

\end{description}\end{quote}
\index{global\_std() (openpivgui.PostProcessing.PostProcessing method)@\spxentry{global\_std()}\spxextra{openpivgui.PostProcessing.PostProcessing method}}

\begin{fulllineitems}
\phantomsection\label{\detokenize{postprocessing:openpivgui.PostProcessing.PostProcessing.global_std}}\pysiglinewithargsret{\sphinxbfcode{\sphinxupquote{global\_std}}}{}{}
Filters vectors by a multiple of the standard deviation.


\sphinxstrong{See also:}


\sphinxcode{\sphinxupquote{openpiv.validation.global\_std()}}



\end{fulllineitems}

\index{global\_val() (openpivgui.PostProcessing.PostProcessing method)@\spxentry{global\_val()}\spxextra{openpivgui.PostProcessing.PostProcessing method}}

\begin{fulllineitems}
\phantomsection\label{\detokenize{postprocessing:openpivgui.PostProcessing.PostProcessing.global_val}}\pysiglinewithargsret{\sphinxbfcode{\sphinxupquote{global\_val}}}{}{}
Filter vectors based on a global min\sphinxhyphen{}max threshold.
\begin{description}
\item[{See:}] \leavevmode
openpiv.validation.global\_val()

\end{description}

\end{fulllineitems}

\index{local\_median() (openpivgui.PostProcessing.PostProcessing method)@\spxentry{local\_median()}\spxextra{openpivgui.PostProcessing.PostProcessing method}}

\begin{fulllineitems}
\phantomsection\label{\detokenize{postprocessing:openpivgui.PostProcessing.PostProcessing.local_median}}\pysiglinewithargsret{\sphinxbfcode{\sphinxupquote{local\_median}}}{}{}
Filter vectors based on a local median threshold.


\sphinxstrong{See also:}


\sphinxcode{\sphinxupquote{openpiv.validation.local\_median\_val()}}



\end{fulllineitems}

\index{repl\_outliers() (openpivgui.PostProcessing.PostProcessing method)@\spxentry{repl\_outliers()}\spxextra{openpivgui.PostProcessing.PostProcessing method}}

\begin{fulllineitems}
\phantomsection\label{\detokenize{postprocessing:openpivgui.PostProcessing.PostProcessing.repl_outliers}}\pysiglinewithargsret{\sphinxbfcode{\sphinxupquote{repl\_outliers}}}{}{}
Replace outliers.

\end{fulllineitems}

\index{sig2noise() (openpivgui.PostProcessing.PostProcessing method)@\spxentry{sig2noise()}\spxextra{openpivgui.PostProcessing.PostProcessing method}}

\begin{fulllineitems}
\phantomsection\label{\detokenize{postprocessing:openpivgui.PostProcessing.PostProcessing.sig2noise}}\pysiglinewithargsret{\sphinxbfcode{\sphinxupquote{sig2noise}}}{}{}
Filter vectors based on the signal to noise threshold.
\begin{description}
\item[{See:}] \leavevmode
openpiv.validation.sig2noise\_val()

\end{description}

\end{fulllineitems}

\index{smoothn\_r() (openpivgui.PostProcessing.PostProcessing method)@\spxentry{smoothn\_r()}\spxextra{openpivgui.PostProcessing.PostProcessing method}}

\begin{fulllineitems}
\phantomsection\label{\detokenize{postprocessing:openpivgui.PostProcessing.PostProcessing.smoothn_r}}\pysiglinewithargsret{\sphinxbfcode{\sphinxupquote{smoothn\_r}}}{}{}
Smoothn postprocessing results.

\end{fulllineitems}


\end{fulllineitems}



\section{vec\_plot}
\label{\detokenize{vec_plot:module-openpivgui.vec_plot}}\label{\detokenize{vec_plot:vec-plot}}\label{\detokenize{vec_plot::doc}}\index{module@\spxentry{module}!openpivgui.vec\_plot@\spxentry{openpivgui.vec\_plot}}\index{openpivgui.vec\_plot@\spxentry{openpivgui.vec\_plot}!module@\spxentry{module}}
Plotting vector data.
\index{get\_dim() (in module openpivgui.vec\_plot)@\spxentry{get\_dim()}\spxextra{in module openpivgui.vec\_plot}}

\begin{fulllineitems}
\phantomsection\label{\detokenize{vec_plot:openpivgui.vec_plot.get_dim}}\pysiglinewithargsret{\sphinxcode{\sphinxupquote{openpivgui.vec\_plot.}}\sphinxbfcode{\sphinxupquote{get\_dim}}}{\emph{\DUrole{n}{array}}}{}
Computes dimension of vector data.

Assumes data to be organised as follows (example):
x  y  v\_x v\_y
16 16 4.5 3.2
32 16 4.3 3.1
16 32 4.2 3.5
32 32 4.5 3.2
\begin{quote}\begin{description}
\item[{Parameters}] \leavevmode
\sphinxstyleliteralstrong{\sphinxupquote{array}} (\sphinxstyleliteralemphasis{\sphinxupquote{np.array}}) \textendash{} Flat numpy array.

\item[{Returns}] \leavevmode
Dimension of the vector field (x, y).

\item[{Return type}] \leavevmode
tuple

\end{description}\end{quote}

\end{fulllineitems}

\index{histogram() (in module openpivgui.vec\_plot)@\spxentry{histogram()}\spxextra{in module openpivgui.vec\_plot}}

\begin{fulllineitems}
\phantomsection\label{\detokenize{vec_plot:openpivgui.vec_plot.histogram}}\pysiglinewithargsret{\sphinxcode{\sphinxupquote{openpivgui.vec\_plot.}}\sphinxbfcode{\sphinxupquote{histogram}}}{\emph{\DUrole{n}{fname}}, \emph{\DUrole{n}{figure}}, \emph{\DUrole{n}{quantity}}, \emph{\DUrole{n}{bins}}, \emph{\DUrole{n}{log\_y}}}{}
Plot an histogram.

Plots an histogram of the specified quantity.
\begin{quote}\begin{description}
\item[{Parameters}] \leavevmode\begin{itemize}
\item {} 
\sphinxstyleliteralstrong{\sphinxupquote{fname}} (\sphinxstyleliteralemphasis{\sphinxupquote{str}}) \textendash{} A filename containing vector data.

\item {} 
\sphinxstyleliteralstrong{\sphinxupquote{figure}} (\sphinxstyleliteralemphasis{\sphinxupquote{matplotlib.figure.Figure}}) \textendash{} An (empty) Figure object.

\item {} 
\sphinxstyleliteralstrong{\sphinxupquote{quantity}} (\sphinxstyleliteralemphasis{\sphinxupquote{str}}) \textendash{} Either v (abs v), v\_x (x\sphinxhyphen{}component) or v\_y (y\sphinxhyphen{}component).

\item {} 
\sphinxstyleliteralstrong{\sphinxupquote{bins}} (\sphinxstyleliteralemphasis{\sphinxupquote{int}}) \textendash{} Number of bins (bars) in the histogram.

\item {} 
\sphinxstyleliteralstrong{\sphinxupquote{log\_scale}} (\sphinxstyleliteralemphasis{\sphinxupquote{boolean}}) \textendash{} Use logaritmic vertical axis.

\end{itemize}

\end{description}\end{quote}

\end{fulllineitems}

\index{pandas\_plot() (in module openpivgui.vec\_plot)@\spxentry{pandas\_plot()}\spxextra{in module openpivgui.vec\_plot}}

\begin{fulllineitems}
\phantomsection\label{\detokenize{vec_plot:openpivgui.vec_plot.pandas_plot}}\pysiglinewithargsret{\sphinxcode{\sphinxupquote{openpivgui.vec\_plot.}}\sphinxbfcode{\sphinxupquote{pandas\_plot}}}{\emph{\DUrole{n}{data}}, \emph{\DUrole{n}{parameter}}, \emph{\DUrole{n}{figure}}}{}
Display a plot with the pandas plot utility.
\begin{quote}\begin{description}
\item[{Parameters}] \leavevmode\begin{itemize}
\item {} 
\sphinxstyleliteralstrong{\sphinxupquote{data}} (\sphinxstyleliteralemphasis{\sphinxupquote{pandas.DataFrame}}) \textendash{} Data to plot.

\item {} 
\sphinxstyleliteralstrong{\sphinxupquote{parameter}} (\sphinxstyleliteralemphasis{\sphinxupquote{openpivgui.OpenPivParams}}) \textendash{} Parameter\sphinxhyphen{}object.

\item {} 
\sphinxstyleliteralstrong{\sphinxupquote{figure}} (\sphinxstyleliteralemphasis{\sphinxupquote{matplotlib.figure.Figure}}) \textendash{} An (empty) figure.

\end{itemize}

\item[{Returns}] \leavevmode


\item[{Return type}] \leavevmode
None.

\end{description}\end{quote}

\end{fulllineitems}

\index{profiles() (in module openpivgui.vec\_plot)@\spxentry{profiles()}\spxextra{in module openpivgui.vec\_plot}}

\begin{fulllineitems}
\phantomsection\label{\detokenize{vec_plot:openpivgui.vec_plot.profiles}}\pysiglinewithargsret{\sphinxcode{\sphinxupquote{openpivgui.vec\_plot.}}\sphinxbfcode{\sphinxupquote{profiles}}}{\emph{\DUrole{n}{fname}}, \emph{\DUrole{n}{figure}}, \emph{\DUrole{n}{orientation}}}{}
Plot velocity profiles.

Line plots of the velocity component specified.
\begin{quote}\begin{description}
\item[{Parameters}] \leavevmode\begin{itemize}
\item {} 
\sphinxstyleliteralstrong{\sphinxupquote{fname}} (\sphinxstyleliteralemphasis{\sphinxupquote{str}}) \textendash{} A filename containing vector data.

\item {} 
\sphinxstyleliteralstrong{\sphinxupquote{figure}} (\sphinxstyleliteralemphasis{\sphinxupquote{matplotlib.figure.Figure}}) \textendash{} An (empty) Figure object.

\item {} 
\sphinxstyleliteralstrong{\sphinxupquote{orientation}} (\sphinxstyleliteralemphasis{\sphinxupquote{str}}) \textendash{} horizontal: Plot v\_y over x.
vertical: Plot v\_x over y.

\end{itemize}

\end{description}\end{quote}

\end{fulllineitems}

\index{scatter() (in module openpivgui.vec\_plot)@\spxentry{scatter()}\spxextra{in module openpivgui.vec\_plot}}

\begin{fulllineitems}
\phantomsection\label{\detokenize{vec_plot:openpivgui.vec_plot.scatter}}\pysiglinewithargsret{\sphinxcode{\sphinxupquote{openpivgui.vec\_plot.}}\sphinxbfcode{\sphinxupquote{scatter}}}{\emph{\DUrole{n}{fname}}, \emph{\DUrole{n}{figure}}}{}
Scatter plot.

Plots v\_y over v\_x.
\begin{quote}\begin{description}
\item[{Parameters}] \leavevmode\begin{itemize}
\item {} 
\sphinxstyleliteralstrong{\sphinxupquote{fname}} (\sphinxstyleliteralemphasis{\sphinxupquote{str}}) \textendash{} Name of a file containing vector data.

\item {} 
\sphinxstyleliteralstrong{\sphinxupquote{figure}} (\sphinxstyleliteralemphasis{\sphinxupquote{matplotlib.figure.Figure}}) \textendash{} An (empty) Figure object.

\end{itemize}

\end{description}\end{quote}

\end{fulllineitems}

\index{vector() (in module openpivgui.vec\_plot)@\spxentry{vector()}\spxextra{in module openpivgui.vec\_plot}}

\begin{fulllineitems}
\phantomsection\label{\detokenize{vec_plot:openpivgui.vec_plot.vector}}\pysiglinewithargsret{\sphinxcode{\sphinxupquote{openpivgui.vec\_plot.}}\sphinxbfcode{\sphinxupquote{vector}}}{\emph{\DUrole{n}{fname}}, \emph{\DUrole{n}{figure}}, \emph{\DUrole{n}{invert\_yaxis}\DUrole{o}{=}\DUrole{default_value}{True}}, \emph{\DUrole{n}{valid\_color}\DUrole{o}{=}\DUrole{default_value}{\textquotesingle{}blue\textquotesingle{}}}, \emph{\DUrole{n}{invalid\_color}\DUrole{o}{=}\DUrole{default_value}{\textquotesingle{}red\textquotesingle{}}}, \emph{\DUrole{o}{**}\DUrole{n}{kw}}}{}
Display a vector plot.
\begin{quote}\begin{description}
\item[{Parameters}] \leavevmode\begin{itemize}
\item {} 
\sphinxstyleliteralstrong{\sphinxupquote{fname}} (\sphinxstyleliteralemphasis{\sphinxupquote{str}}) \textendash{} Pathname of a text file containing vector data.

\item {} 
\sphinxstyleliteralstrong{\sphinxupquote{figure}} (\sphinxstyleliteralemphasis{\sphinxupquote{matplotlib.figure.Figure}}) \textendash{} An (empty) Figure object.

\end{itemize}

\end{description}\end{quote}

\end{fulllineitems}



\section{CreateToolTip}
\label{\detokenize{createtooltip:module-openpivgui.CreateToolTip}}\label{\detokenize{createtooltip:createtooltip}}\label{\detokenize{createtooltip::doc}}\index{module@\spxentry{module}!openpivgui.CreateToolTip@\spxentry{openpivgui.CreateToolTip}}\index{openpivgui.CreateToolTip@\spxentry{openpivgui.CreateToolTip}!module@\spxentry{module}}
Tooltips for tkinter widgets.
\index{CreateToolTip (class in openpivgui.CreateToolTip)@\spxentry{CreateToolTip}\spxextra{class in openpivgui.CreateToolTip}}

\begin{fulllineitems}
\phantomsection\label{\detokenize{createtooltip:openpivgui.CreateToolTip.CreateToolTip}}\pysiglinewithargsret{\sphinxbfcode{\sphinxupquote{class }}\sphinxcode{\sphinxupquote{openpivgui.CreateToolTip.}}\sphinxbfcode{\sphinxupquote{CreateToolTip}}}{\emph{\DUrole{n}{widget}}, \emph{\DUrole{n}{text}\DUrole{o}{=}\DUrole{default_value}{\textquotesingle{}No tooltip available.\textquotesingle{}}}}{}
Create a tooltip for a given widget as the mouse goes on it.

See \sphinxurl{https://stackoverflow.com/a/3222120} for original authors and
improved versions.
\begin{quote}\begin{description}
\item[{Parameters}] \leavevmode\begin{itemize}
\item {} 
\sphinxstyleliteralstrong{\sphinxupquote{widget}} (\sphinxstyleliteralemphasis{\sphinxupquote{tkinter.widget}}) \textendash{} A tkinter widget object.

\item {} 
\sphinxstyleliteralstrong{\sphinxupquote{text}} (\sphinxstyleliteralemphasis{\sphinxupquote{str}}) \textendash{} A tooltip text.

\item {} 
\sphinxstyleliteralstrong{\sphinxupquote{widget}} \textendash{} A tkinter widget object.

\item {} 
\sphinxstyleliteralstrong{\sphinxupquote{text}} \textendash{} A tooltip text.

\end{itemize}

\end{description}\end{quote}

\end{fulllineitems}



\section{open\_piv\_gui\_tools}
\label{\detokenize{open_piv_gui_tools:module-openpivgui.open_piv_gui_tools}}\label{\detokenize{open_piv_gui_tools:open-piv-gui-tools}}\label{\detokenize{open_piv_gui_tools::doc}}\index{module@\spxentry{module}!openpivgui.open\_piv\_gui\_tools@\spxentry{openpivgui.open\_piv\_gui\_tools}}\index{openpivgui.open\_piv\_gui\_tools@\spxentry{openpivgui.open\_piv\_gui\_tools}!module@\spxentry{module}}
Methods for reuse within the OpenPivGui project.
\index{create\_save\_vec\_fname() (in module openpivgui.open\_piv\_gui\_tools)@\spxentry{create\_save\_vec\_fname()}\spxextra{in module openpivgui.open\_piv\_gui\_tools}}

\begin{fulllineitems}
\phantomsection\label{\detokenize{open_piv_gui_tools:openpivgui.open_piv_gui_tools.create_save_vec_fname}}\pysiglinewithargsret{\sphinxcode{\sphinxupquote{openpivgui.open\_piv\_gui\_tools.}}\sphinxbfcode{\sphinxupquote{create\_save\_vec\_fname}}}{\emph{\DUrole{n}{path}\DUrole{o}{=}\DUrole{default_value}{\textquotesingle{}/home/peter/fhms/Bin/py/openpiv\sphinxhyphen{}tk\sphinxhyphen{}gui/docs\textquotesingle{}}}, \emph{\DUrole{n}{basename}\DUrole{o}{=}\DUrole{default_value}{None}}, \emph{\DUrole{n}{postfix}\DUrole{o}{=}\DUrole{default_value}{\textquotesingle{}\textquotesingle{}}}, \emph{\DUrole{n}{count}\DUrole{o}{=}\DUrole{default_value}{\sphinxhyphen{} 1}}, \emph{\DUrole{n}{max\_count}\DUrole{o}{=}\DUrole{default_value}{9}}}{}
Assembles a valid absolute path for saving vector data.
\begin{quote}\begin{description}
\item[{Parameters}] \leavevmode\begin{itemize}
\item {} 
\sphinxstyleliteralstrong{\sphinxupquote{path}} (\sphinxstyleliteralemphasis{\sphinxupquote{str}}) \textendash{} Directory path. Default: Working directory.

\item {} 
\sphinxstyleliteralstrong{\sphinxupquote{basename}} (\sphinxstyleliteralemphasis{\sphinxupquote{str}}) \textendash{} Prefix. Default: None.

\item {} 
\sphinxstyleliteralstrong{\sphinxupquote{postfix}} (\sphinxstyleliteralemphasis{\sphinxupquote{str}}) \textendash{} Postfix. Default: None.

\item {} 
\sphinxstyleliteralstrong{\sphinxupquote{count}} (\sphinxstyleliteralemphasis{\sphinxupquote{int}}) \textendash{} Counter for numbering filenames.
Default: \sphinxhyphen{}1 (no number)

\item {} 
\sphinxstyleliteralstrong{\sphinxupquote{max\_count}} (\sphinxstyleliteralemphasis{\sphinxupquote{int}}) \textendash{} Highest number to expect. Used for generating
leading zeros. Default: 9 (no leading zeros).

\end{itemize}

\end{description}\end{quote}

\end{fulllineitems}

\index{get\_dim() (in module openpivgui.open\_piv\_gui\_tools)@\spxentry{get\_dim()}\spxextra{in module openpivgui.open\_piv\_gui\_tools}}

\begin{fulllineitems}
\phantomsection\label{\detokenize{open_piv_gui_tools:openpivgui.open_piv_gui_tools.get_dim}}\pysiglinewithargsret{\sphinxcode{\sphinxupquote{openpivgui.open\_piv\_gui\_tools.}}\sphinxbfcode{\sphinxupquote{get\_dim}}}{\emph{\DUrole{n}{array}}}{}
Computes dimension of vector data.

Assumes data to be organised as follows (example):
x  y  v\_x v\_y
16 16 4.5 3.2
32 16 4.3 3.1
16 32 4.2 3.5
32 32 4.5 3.2
\begin{quote}\begin{description}
\item[{Parameters}] \leavevmode
\sphinxstyleliteralstrong{\sphinxupquote{array}} (\sphinxstyleliteralemphasis{\sphinxupquote{np.array}}) \textendash{} Flat numpy array.

\item[{Returns}] \leavevmode
Dimension of the vector field (x, y).

\item[{Return type}] \leavevmode
tuple

\end{description}\end{quote}

\end{fulllineitems}

\index{str2dict() (in module openpivgui.open\_piv\_gui\_tools)@\spxentry{str2dict()}\spxextra{in module openpivgui.open\_piv\_gui\_tools}}

\begin{fulllineitems}
\phantomsection\label{\detokenize{open_piv_gui_tools:openpivgui.open_piv_gui_tools.str2dict}}\pysiglinewithargsret{\sphinxcode{\sphinxupquote{openpivgui.open\_piv\_gui\_tools.}}\sphinxbfcode{\sphinxupquote{str2dict}}}{\emph{\DUrole{n}{s}}}{}
Parses a string representation of a dictionary.
\begin{quote}\begin{description}
\item[{Parameters}] \leavevmode
\sphinxstyleliteralstrong{\sphinxupquote{s}} (\sphinxstyleliteralemphasis{\sphinxupquote{str}}) \textendash{} Comma separated list of colon separated key value pairs.

\end{description}\end{quote}
\subsubsection*{Example}

str2dict(‘key1: value1’, ‘key2: value2’)

\end{fulllineitems}

\index{str2list() (in module openpivgui.open\_piv\_gui\_tools)@\spxentry{str2list()}\spxextra{in module openpivgui.open\_piv\_gui\_tools}}

\begin{fulllineitems}
\phantomsection\label{\detokenize{open_piv_gui_tools:openpivgui.open_piv_gui_tools.str2list}}\pysiglinewithargsret{\sphinxcode{\sphinxupquote{openpivgui.open\_piv\_gui\_tools.}}\sphinxbfcode{\sphinxupquote{str2list}}}{\emph{\DUrole{n}{s}}}{}
Parses a string representation of a list.
\begin{quote}\begin{description}
\item[{Parameters}] \leavevmode
\sphinxstyleliteralstrong{\sphinxupquote{s}} (\sphinxstyleliteralemphasis{\sphinxupquote{str}}) \textendash{} String containing comma separated values.

\end{description}\end{quote}
\subsubsection*{Example}

str2list(‘img01.png’, ‘img02.png’)
\begin{quote}\begin{description}
\item[{Returns}] \leavevmode


\item[{Return type}] \leavevmode
list

\end{description}\end{quote}

\end{fulllineitems}



\chapter{Related}
\label{\detokenize{related:related}}\label{\detokenize{related::doc}}
Also check out \sphinxhref{https://eguvep.github.io/jpiv}{JPIV}, our Java based PIV evaluation suite.


\chapter{Indices and tables}
\label{\detokenize{index:indices-and-tables}}\begin{itemize}
\item {} 
\DUrole{xref,std,std-ref}{genindex}

\item {} 
\DUrole{xref,std,std-ref}{modindex}

\item {} 
\DUrole{xref,std,std-ref}{search}

\end{itemize}


\renewcommand{\indexname}{Python Module Index}
\begin{sphinxtheindex}
\let\bigletter\sphinxstyleindexlettergroup
\bigletter{o}
\item\relax\sphinxstyleindexentry{openpivgui.CreateToolTip}\sphinxstyleindexpageref{createtooltip:\detokenize{module-openpivgui.CreateToolTip}}
\item\relax\sphinxstyleindexentry{openpivgui.MultiProcessing}\sphinxstyleindexpageref{multiprocessing:\detokenize{module-openpivgui.MultiProcessing}}
\item\relax\sphinxstyleindexentry{openpivgui.open\_piv\_gui\_tools}\sphinxstyleindexpageref{open_piv_gui_tools:\detokenize{module-openpivgui.open_piv_gui_tools}}
\item\relax\sphinxstyleindexentry{openpivgui.OpenPivGui}\sphinxstyleindexpageref{openpivgui:\detokenize{module-openpivgui.OpenPivGui}}
\item\relax\sphinxstyleindexentry{openpivgui.OpenPivParams}\sphinxstyleindexpageref{openpivparams:\detokenize{module-openpivgui.OpenPivParams}}
\item\relax\sphinxstyleindexentry{openpivgui.PostProcessing}\sphinxstyleindexpageref{postprocessing:\detokenize{module-openpivgui.PostProcessing}}
\item\relax\sphinxstyleindexentry{openpivgui.vec\_plot}\sphinxstyleindexpageref{vec_plot:\detokenize{module-openpivgui.vec_plot}}
\end{sphinxtheindex}

\renewcommand{\indexname}{Index}
\printindex
\end{document}